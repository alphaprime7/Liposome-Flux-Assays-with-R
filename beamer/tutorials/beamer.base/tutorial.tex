\documentclass{beamer}
\usepackage{multicol}
%\usetheme{metropolis}
%\usepackage{appendixnumberbeamer}

%\usepackage{booktabs}
%\usepackage[scale=2]{ccicons}

%\usepackage{pgfplots}
%\usepgfplotslibrary{dateplot}

%\usepackage{xspace}
%\newcommand{\themename}{\textbf{\textsc{metropolis}}\xspace}
%\documentclass[handout]{beamer}
%\usepackage{pgfpages}
%\pgfpagesuselayout{4 on 1}[a4paper,border shrink=5mm,landscape]
\usetheme[progressbar=frametitle]{metropolis}
%\setbeamertemplate{frame numbering}{fraction}
\useoutertheme{metropolis}
\useinnertheme{metropolis}
\usefonttheme{metropolis}
\usecolortheme{spruce}
\setbeamercolor{background canvas}{bg=white}
%\usetheme{Warsaw}
\definecolor{mygreen}{rgb}{.125,1.0,.25}
\usecolortheme[named=mygreen]{structure}
%%\usecolortheme{crane}
%\useinnertheme{rectangles}
%\useoutertheme{tree}
%\usefonttheme{serif}


\title{Understanding Liposome Flux Assays in the context of Sodium voltage-gated channels}
\subtitle{my first beamer tutorial}
\author{Tingwei Adeck}
\institute{West Texas A \& M University}
\date{\today}

\begin{document}

\begin{frame}
\titlepage
\end{frame}

\begin{frame}
\label{contents}
\frametitle{Outline}
\tableofcontents
\end{frame}

\section{Section 1}
\subsection{sub a}

\begin{frame}
\label{movies}
\frametitle{Best Movies 2019}
\begin{itemize}
\item Avengers Endgame
\pause
\item Dark Phoenix
\pause
\item Spider Man into the Spider Verse
\pause
\item Escape Plan
\pause
\item Will remember more
\end{itemize}
\end{frame}

\begin{frame}[fragile]
\label{codes}
\frametitle{Including Code}
\begin{semiverbatim}
\\begin\{frame\}
\\frametitle\{Outline\}
\\tableofcontents
\\end\{frame\}
\end{semiverbatim}
\end{frame}

\begin{frame}
\frametitle{Title}
Lorem ipsum bs
\end{frame}

\subsection{sub b}
\begin{frame}
\frametitle{Using Columns}
\begin{columns}
\column{0.5\textwidth}
Lorem ipsum bs
\column{0.5\textwidth}
\centering
\includegraphics[scale=0.2]{cover.jpg}
\end{columns}
\end{frame}

\begin{frame}
\frametitle{Pictures}
\begin{figure}
\includegraphics[scale=0.4]{cover.jpg}
\caption{Typewriting in Java Source}
\end{figure}
\end{frame}

\begin{frame}
\frametitle{Descriptions}
\begin{description}
\item[API] Application Programming Interface
\item[LAN] Local Area Network
\item[ASCI] American Standard Code for Information Interchange
\end{description}
\end{frame}

\setbeamercovered{invisible}
\begin{frame}
\frametitle{Tables}
\begin{table}
\begin{tabular}{l | c | c | c | c }
Competitor Name & Swim & Cycle & Run & Total \\
\hline \hline
John T & 13:04 & 24:15 & 18:34 & 55:53 \onslide<2->\\ 
Norman P & 8:00 & 22:45 & 23:02 & 53:47 \onslide<3->\\
Alex K & 14:00 & 28:00 & n/a & n/a \onslide<4->\\ 
Sarah H & 9:22 & 21:10 & 24:03 & 54:35 
\end{tabular}
\caption{Triathlon Results}
\end{table}
\end{frame}

\begin{frame}
\frametitle{Blocks}
\begin{block}{Block Title}
Getting used to the feeling
\end{block}
\begin{alertblock}{Block Title}
Getting used to something new and warning signs
\end{alertblock}
\end{frame}

\begin{frame}
\frametitle{More Blocks}
\begin{definition}
A prime number is a number with only one factor and itself.
\end{definition}
\begin{example}
A prime number example is 2.
\end{example}
\end{frame}

\begin{frame}
\frametitle{Math Blocks}
\begin{theorem}<1->{Pythagoras}
$ a^2 + b^2 = c^2$
\end{theorem}
\begin{corollary}<3->
$ x + y = y + x  $
\end{corollary}
\begin{proof}<2->
$\omega +\phi = \epsilon $
\end{proof}
\end{frame}

\begin{frame}
\frametitle{buttons}
\hyperlink{contents}{\beamerbutton{Contents Page}}

\hyperlink{codes}{\beamergotobutton{Codes}}

\hyperlink{movies}{\beamerskipbutton{Movies}}

\hyperlink{movies}{\beamerreturnbutton{Movies}}
\end{frame}

\begin{frame}
\frametitle{Overlays}\vspace{2pt}
\onslide<1->{first line of text}

\onslide<2->{second line of text}

\onslide<3->{third line of text}
\end{frame}

\setbeamercovered{transparent}
\begin{frame}
\frametitle{Overlays}
\uncover<1->{first line of text}

\uncover<2->{second line of text}

\uncover<3->{third line of text}
\end{frame}

\begin{frame}
\frametitle{compatible commands}
\textbf<2>{example text}

\textsl<3>{example text}

\textit<4>{example text}

\textrm<5>{example text}

\textsf<6>{example text}

\textcolor<7>{orange}{example text}

\alert<8>{example text}

\structure<9>{example text}
\end{frame}

\begin{frame}[t]{Title of the slide}\vspace{0.5pt}
\begin{enumerate}
\item fuck you bammel
\end{enumerate}
\end{frame}

\begin{frame}[t]{Functions}\vspace{4pt}
\begin{block}{Definition of a Function}
\vspace{0.5em}
A \textbf{function} is $f$ is a rule that assigns to each element $x$ in a set $D$ exactly one element, called $f(x)$, in a set $E$.
\vspace{0.5em}
\end{block}

\vspace{10pt}
Set $D$ is callde the 
\only<1>{\line(1,0){50}}
\only<2>{\textcolor{magenta}{domain}}
\,of the function $E$.\\[10pt]

\end{frame}

\begin{frame}[t]{Multicol practice}\vspace{4pt}
you should be able to identify these by name
\begin{enumerate}
\begin{multicols}{3}
\item $y=x$
\item $y=|x|$
\item $y=x^2$
\item $y=x^3$
\item $y=\sqrt{x}$
\item $y=\sqrt[3]{x}$
\item $y=\frac{1}{x}$
\end{multicols}
\end{enumerate}

\end{frame}

\end{document}